Before reading this, please read \Section{sec:erlang-and-otp}.
\Paragraph{subsub:CodingConventions} gives order to get an overview on
coding conventions: the last parameter of all events is the \emph{status}
of the running logic.

\subsection{The \Keyword{keeper} \Acr{API}}

    \subsubsection{Events and callbacks}

        \begin{itemize}
        \item \Func{init}{1} \\
            This function is provided with an extern parameter,
            corresponding to the \Const{keeper_args} voice in the
            configuration file (\see{\Subsection{sub:ConfigurationFile}}).
            It is supposed to do some initialization (for instance
            spawning peers using the \Func{keeper_proto:add\_peers}{3}).
            It can return \RetOK{Status} in case of success or
            \RetErr{Reason} in case of failure;

        \item \Func{handle_spawn_notify}{2} \\
            A freshly peer can call the \Func{peer_ctrl:notify_spawn}{0}
            function. This will result in this event to be issued. The
            first parameter is the \Keyword{Pid} of the notifying process;

        \item \Func{handle_result_notify}{3} \\
            A peer can call the \Func{peer_ctrl:notify_result}{1}
            function in order to send a piece of elaborated information to
            the \Keyword{keeper}. This will result in this event to b e
            issued. The first parameter is the \emph{pid} of the
            peer, the second is the result;

        \item \Func{handle_term_notify}{4} \\
            When a peer dies (both for normal termination or crash) the
            \Keyword{keeper} gets notified trough this event. The first
            parameter is the \emph{pid} of the peer, the second may be the
            \Const{undefined} atom or a \emph{\Erlang\ reference}
            associated to the spawning (\see{\Erlang\ documentation}), the
            third is the reason of the termination;

        \item \Func{handle_info}{2} \\
            If the \Keyword{abstract component} gets some raw message, not
            appropriately enveloped with the underlaying protocol, the
            \Keyword{keeper} gets notified through this event. The first
            parameter is, of course, the raw packet.

        \end{itemize}

    \subsubsection{Library functions}

        \begin{enumerate}
        \item   \Const{keeper\_inject}

            \begin{itemize}

            \item \Func{send}{2} \\
                Using this function, the keeper can send a message to a
                specific node (first parameter), triggering its
                \Func{handle_introduction}{3} callback. The message
                (second parameter) is supposed to be understood by the
                \Keyword{specific peer}, which will see the \emph{atom}
                \Keyword{keeper} as sender;

            \item \Func{send}{3} \\
                This function is the same as \Func{send}{2}, except it has
                and additional parameter (the first one) which specifies a
                sender for the message. This can be used, for instance, to
                spoof a \emph{pid};

            \item \Func{introduce}{2} \\
                This function triggers the \Func{handle_introduction}{3}
                callback of the peer which \emph{pid} is specified as
                first parameter. The peer will be introduced to the peer
                specified as second parameter. The sender of the
                introduction message will be the \emph{atom}
                \Keyword{keeper};

            \item \Func{introduce}{3} \\
                This function is the same as \Func{introduce}{2}, except
                it has and additional parameter (the first one) which
                specifies a sender for the introduction message. This can
                be used, for instance, to spoof a \emph{pid};

            \item \Func{bcast}{1} \\
                This function achieves broadcasting of a message. It's
                pretty similar to \Func{send}{2}, except of course it
                doesn't require a target \emph{pid};

            \item \Func{bcast}{2} \\
                This function achieves broadcasting of a message. It's
                pretty similar to \Func{send}{3}, except of course it
                doesn't require a target \emph{pid};

            \end{itemize}

        \item   \Const{keeper\_proto}

            \begin{itemize}
            \item \Func{add_peers}{3}
            \item \Func{add_peers_pidonly}{3}
            \item \Func{enable_beacon}{1}
            \item \Func{disable_beacon}{0}
            \end{itemize}

        \end{enumerate}

\subsection{The \Keyword{peer} \Acr{API}}

    \subsubsection{Events and callbacks}
        \begin{itemize}
        \item \Func{init}{1}
        \item \Func{handle_message}{3}
        \item \Func{handle_introduction}{3}
        \item \Func{handle_beacon}{1}
        \item \Func{handle_info}{2}
        \end{itemize}

    \subsubsection{Library functions}

        \begin{enumerate}
            \item   \Const{peer\_chan}
                \begin{itemize}
                \item \Func{send}{2}
                \item \Func{greet}{1}
                \item \Func{bcast_send}{1}
                \item \Func{bcast_greet}{0}
                \end{itemize}

            \item   \Const{peer\_ctrl}
                \begin{itemize}
                \item \Func{notify_spawn}{0}
                \item \Func{notify_result}{1}
                \item \Func{notify_term}{1}
                \item \Func{notify_term}{0}
                \end{itemize}
        \end{enumerate}

\subsection{The configuration file} \label{sub:ConfigurationFile}


