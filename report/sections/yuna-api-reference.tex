Before reading this, please read \Section{sec:erlang-and-otp}.
\Paragraph{subsub:CodingConventions} gives order to get an overview on
coding conventions: the last parameter of all \emph{events} is the
\emph{status} of the running logic.

% -----------------------------------------------------------------------
\subsection{The \Keyword{keeper} \Acr{API}}
% -----------------------------------------------------------------------

    \subsubsection{Events and callbacks}

        \begin{itemize}
        \item \Func{init}{1} \\
            This function is provided with an extern parameter,
            corresponding to the \Const{keeper_args} voice in the
            configuration file (\see{\Subsection{sub:ConfigurationFile}}).
            It is supposed to do some initialization (for instance
            spawning peers using the \Func{keeper_proto:add\_peers}{3}).
            It can return \RetOK{Status} in case of success or
            \RetErr{Reason} in case of failure;

        \item \Func{handle_spawn_notify}{2} \\
            A freshly peer can call the \Func{peer_ctrl:notify_spawn}{0}
            function. This will result in this event to be issued. The
            first parameter is the \Keyword{Pid} of the notifying process;

        \item \Func{handle_result_notify}{3} \\
            A peer can call the \Func{peer_ctrl:notify_result}{1}
            function in order to send a piece of elaborated information to
            the \Keyword{keeper}. This will result in this event to b e
            issued. The first parameter is the \emph{pid} of the
            peer, the second is the result;

        \item \Func{handle_term_notify}{4} \\
            When a peer dies (both for normal termination or crash) the
            \Keyword{keeper} gets notified trough this event. The first
            parameter is the \emph{pid} of the peer, the second may be the
            \Const{undefined} atom or a \emph{\Erlang\ reference}
            associated to the spawning (\see{\Erlang\ documentation}), the
            third is the reason of the termination;

        \item \Func{handle_info}{2} \\
            If the \Keyword{abstract component} gets some raw message, not
            appropriately enveloped with the underlaying protocol, the
            \Keyword{keeper} gets notified through this event. The first
            parameter is, of course, the raw packet.

        \end{itemize}

    \subsubsection{Library functions}

        \begin{enumerate}
        \item   \Const{keeper\_inject}

            \begin{itemize}

            \item \Func{send}{2} \\
                Using this function, the keeper can send a message to a
                specific node (first parameter), triggering its
                \Func{handle_introduction}{3} callback. The message
                (second parameter) is supposed to be understood by the
                \Keyword{specific peer}, which will see the \emph{atom}
                \Const{keeper} as sender;

            \item \Func{send}{3} \\
                This function is the same as \Func{send}{2}, except it has
                and additional parameter (the first one) which specifies a
                sender for the message. This can be used, for instance, to
                spoof a \emph{pid};

            \item \Func{introduce}{2} \\
                This function triggers the \Func{handle_introduction}{3}
                callback of the peer which \emph{pid} is specified as
                first parameter. The peer will be introduced to the peer
                specified as second parameter. The sender of the
                introduction message will be the \emph{atom}
                \Const{keeper};

            \item \Func{introduce}{3} \\
                This function is the same as \Func{introduce}{2}, except
                it has and additional parameter (the first one) which
                specifies a sender for the introduction message. This can
                be used, for instance, to spoof a \emph{pid};

            \item \Func{bcast}{1} \\
                This function achieves broadcasting of a message. It's
                pretty similar to \Func{send}{2}, except of course it
                doesn't require a target \emph{pid};

            \item \Func{bcast}{2} \\
                This function achieves broadcasting of a message. It's
                pretty similar to \Func{send}{3}, except of course it
                doesn't require a target \emph{pid};

            \end{itemize}

        \item   \Const{keeper_proto}

            \begin{itemize}
            \item \Func{add_peers}{3} \\
                Insert new nodes in the \Const{peers}
                \Keyword{supervisor}. The parameters are the number $N$ of
                nodes to spawn, an \emph{atom} giving the name of the
                \Keyword{specific component} which implements the peer
                logic, and finally the parameter for the logic (which will
                be passed as first parameter for the \Func{init}{1}
                function of the specified module). \\
                Note that this function can be called multiple times,
                adding new nodes to the running protocol. The nodes can be
                added at any time, and may possibly correspond to
                different \Keyword{specific components}. \\
                The function yields a list of tuples \Const{\{Pid, Ref\}},
                where \Const{Pid} is the process identifier of the spawned
                process, and \Const{Ref} is the reference of an
                \emph{\Erlang\ monitor}, which can be used to match the
                second parameter of the \Func{handle_term_nofity}{4}
                event;

            \item \Func{add_peers_pidonly}{3} \\
                Same as \Func{add_peers}{3}, just returns a list of
                process identifiers, discarding \emph{references};

            \item \Func{enable_beacon}{1} \\
                Require the \Const{bcast} component to send periodically a
                beacon to the subscribers. The parameter to be provided is
                the beacon period (in milliseconds);

            \item \Func{disable_beacon}{0} \\
                Disable the beacon enabled with \Func{enable_beacon}{0}.

            \end{itemize}

        \end{enumerate}

% -----------------------------------------------------------------------
\subsection{The \Keyword{peer} \Acr{API}}
% -----------------------------------------------------------------------

    \subsubsection{Events and callbacks}
        \begin{itemize}
        \item \Func{init}{1}
            This event is provided with an extern parameter,
            corresponding to the last parameter of the
            \Func{keeper_proto:add_peers}{3} function.  It is supposed to
            do some initialization dependent on the \Keyword{specific
            component} implementing the peer.  It can return
            \RetOK{Status} in case of success or \RetErr{Reason} in
            case of failure;

        \item \Func{handle_message}{3} \\
            This event is called when the \Keyword{generic component} is
            messaged with a valid message. The first parameter corresponds
            to the process identifier of the sender or to the \emph{atom}
            \Const{keeper} if the message has been send by the
            \Keyword{keeper} through the \Func{keeper_inject:send}{2}

        \item \Func{handle_introduction}{3} \\
            This event is triggered when the \Keyword{peer}
            has been introduced to another \Keyword{peer}. As this
            introduction can be done both by the \Keyword{keeper} or by
            another \Keyword{peer}, the first argument can be either the
            \Const{keeper} \emph{atom} or a \emph{pid}.

        \item \Func{handle_beacon}{1} \\
            If the \Keyword{keeper} enables the beacon (trough the
            \Func{keeper_proto:enable_beacon}{1} function), this event
            will be called periodically.

        \item \Func{handle_info}{2} \\
             If the \Keyword{abstract component} gets some raw message, not
            appropriately enveloped with the underlaying protocol, the
            \Keyword{peer} gets notified through this event. The first
            parameter is, of course, the raw packet.

        \end{itemize}

    \subsubsection{Library functions}

        \begin{enumerate}
            \item   \Const{peer\_chan}
                \begin{itemize}
                \item \Func{send}{2} \\
                    Send a message through the internal protocol. The
                    first parameter is the \emph{pid} of the target
                    process, the second is the message;

                \item \Func{greet}{1} \\
                    Greet a \Keyword{peer} by raising its
                    \Func{handle_introduction}{3} event. The first
                    parameter is the \emph{pid} of the target process;

                \item \Func{bcast_send}{1} \\
                    Broadcast a message through the internal protocol. The
                    parameter is the message to broadcast;

                \item \Func{bcast_greet}{0} \\
                    Greet all \Keyword{peers}, in broadcast, raising their
                    \Func{handle_introduction}{3} event.

                \end{itemize}

            \item   \Const{peer\_ctrl}
                \begin{itemize}
                \item \Func{notify_spawn}{0} \\
                    Notify the \Keyword{keeper} about a successfull
                    spawning (see the \Func{handle_spawn_notify}{2}
                    event)

                \item \Func{notify_result}{1} \\
                    Notify the \Keyword{keeper} about a result
                    (see the \Func{handle_result_notify}{3});

                \item \Func{notify_term}{1} \\
                    Notify the \Keyword{keeper} about process termination.
                    Note that, as this function is used internally
                    already, there's no need to use it explicitly
                    (see the \Func{handle_term_notify}{4});

                \item \Func{notify_term}{0} \\
                    As \Const{notify_term(normal)}.

                \end{itemize}
        \end{enumerate}


% -----------------------------------------------------------------------
\subsection{The configuration file} \label{sub:ConfigurationFile}
% -----------------------------------------------------------------------

Basing on the \Erlang\ coding standards, the configuration of the
application is placed in a file named \Path{ebin/$A$.app}, where $A$ is
the name of the application. In our case the file is \Path{ebin/yuna.app},
shown in \Listing{code:ConfFile}.

\begin{figure}[hbt]
\begin{lstlisting}[label={code:ConfFile},
                   caption={default configuration file}]
{application, yuna, [
    {description, "YUNA - Distributed Consensus implementation"},
    {vsn, "0.0.1"},
    {modules, [chan_filters, tweaked_chan, bcast, peers, keeper_proto,
               keeper_inject, yuna, gen_peer, gen_keeper, peer_chan,
               peer_ctrl, main, log_serv, randel, utils, services]}
    {registered, []},
    {application, [kernel, stdlib]},
    {mod, {yuna, []}},

    {env, [
        % Default configuration, can be overriden by configuration file.
        {faulty_prob, 0.01},        % 1% of nodes are faulty
        {faulty_fail_prob, 0.05},   % 5% crash probab. for faulty node

        {deliver_mindel, 500},      % Minimum deliver delay
        {deliver_maxdel, 1500},     % Maximum deliver delay
        {deliver_dist, {random, uniform, []}},  % Delay distribution

        {log_args, { standard_error,    % For visual logging
                     "statfile"         % For stats retrival
                   }
        },
        {keeper, gfd_keeper},
        {keeper_args, { {10, 20, 4},     % TFail, TCleanup, TGossip
                        500,            % NPeers
                        true,           % Don't wait for user modifications
                        0.01,           % StatPeers [ratio]
                        500,            % TBeacon, [ms]
                        120             % BeaconWait, [TBeacon]
                        % 120 beacons = 1 min
                      }
        }
    ]}
]}.
\end{lstlisting}
\end{figure}

The first chunk of the file describes the application details (giving
the name of the application, the version number and so on), while the
remaining part of the file contains the environment definition as
\emph{key-value} mapping. The accepted format matches the following
pattern:

\begin{quote}
\centering
\Const{\{env, [\{K1, V1\}, \{K2, V2\} ... \{Kn, Vn\}]\}}
\end{quote}

The configuration in the \Path{.app} file can be overriden by another
configuration in a \Path{.conf} file by passing it as parameter to the
\Erlang\ interpreter: suppose our file is called \Path{override.conf}, we
can do the following:
\begin{quote}
\begin{verbatim}
erl -conf override
\end{verbatim}
\end{quote}

All this details are better explained in the official \Erlang\
documentation\cite{bib:ErlApp}. Since however this is pretty boring, I've
decided to write a little \emph{perl} script which is capable of
sintetizing a correct configuration file starting from a simple text file
containing the parameters. \See{\Paragraph{subsub:Automagic}}.

\subsubsection{Parameters for the \YUNA\ environment}
\label{subsub:ConfEnvParams}

    (\see{\Subsection{sub:EnvParams}})

    \begin{itemize}

    \item   The probability of having a faulty node \PrName{faulty}
            corresponds to the \Const{faulty_prob} key;

    \item   The probability, for a faulty node, of fail during a
            transmission \PrName{fc} corresponds to the
            \Const{faulty_fail_prob}.

    \item   The minimum delivery delay \SubsVar{\delta}{min} corresponds
            to the \Const{deliver_mindel} key;

    \item   The maximum delivery delay \SubsVar{\delta}{max} corresponds
            to the \Const{deliver_maxdel} key;

    \item   The probability distribution of the delay \SubsVar{D}{delay}
            is specified as an \Erlang\ \emph{tuple} \Const{\{M, F, A\}}
            providing the coordinates (\emph{module}, \emph{function name}
            and \emph{arguments}) of the function. A uniform distribution
            can be obtained using \Const{\{random, uniform, []\}}. This
            parameter corresponds to the \Const{deliver_dist} key.

    \item   Due to the fact that the \Keyword{keeper} is in charge of
            spawning the \Keyword{peers}, the number of nodes $n =
            \left|\Pi\right|$ is part of the \Keyword{keeper}
            configuration, thus described in
            \Paragraph{subsub:ConfKeeperParams};

    \item   The same is valid for the \emph{beacon}, thus
            \SubsVar{T}{beacon} is part of the \Keyword{keeper}
            configuration, described in
            \Paragraph{subsub:ConfKeeperParams} as well.

    \end{itemize}

\subsubsection{Technical parameters}
\label{subsub:ConfTechParams}

    \begin{itemize}

        \item   The only one parameter for the logger is associated to the
                key \Const{log_args}. It must be a tuple \Const{\{Stream,
                NamePrefix\}}, where \Const{Stream} is a file descriptor
                (like \Const{standard_error}, which is the default), and
                \Const{NamePrefix} is a string giving the prefix for the
                logger output file names
                (\see{\Subsection{sub:YunaOverview}});

        \item   The \Keyword{keeper} parametrization and selection goes
                trough two keys:
            \begin{itemize}
                \item   The \Const{keeper} key must be associated with an
                        \emph{atom} declaring the name of the
                        \Keyword{specific componet} to be used as
                        \Keyword{keeper};
                \item   The \Const{keeper_args}
                        key can be associated to any \Erlang\ item, which
                        will be the actual parameter for the
                        \Func{init}{1} function of the \Keyword{specific
                        keeper}.
            \end{itemize}

        \item   We need to tune the system and get the better trade-off
                between statistics significance and overhead.
                \Const{StatPeers} is the ratio $r \in [0, 1]$ of processes
                which will report statistics. It's par of the
                \Keyword{keeper} configuration, thus described in
                \Paragraph{subsub:ConfKeeperParams}.

    \end{itemize}

\subsubsection{Parameters for Consensus}
\label{subsub:ConfConsParams}

    All the parameters for the distributed system nodes
    (\see{\Subsection{sub:ConsParams}}) are managed by the
    \Keyword{keeper} during the spawning phase.
    \Subsection{sub:ConsParams} describes the configurations for
    \SubsVar{T}{fail}, \SubsVar{T}{cleanup} and \SubsVar{T}{gossip}.

\subsubsection{Keeper configuration}
\label{subsub:ConfKeeperParams}

    The \Keyword{specific keeper} for the implemented \emph{Consensus}
    algorithm accepts as parameter a \emph{tuple} of the form
    \begin{quote}
    \centering
    \Const{\{PeerParams, NPeers, StatPeers, TBeacon, BeaconWait\}}
    \end{quote}
    where
    \begin{itemize}
    \item   \Const{PeerParams} contains the parameters for the
            \Keyword{peers}
    \item   \Const{NPeers} is the number $n$ of processes running the
            algorithm;
    \item   \Const{TBeacon} is the beacon time period \SubsVar{T}{beacon}.
    \end{itemize}

    The \Const{PeerParams} is required to be a tuple of the form
    \begin{quote}
    \centering
    \Const{\{TFail, TCleanup, TGossip\}}
    \end{quote}
    respectively corresponding to \SubsVar{T}{fail}, \SubsVar{T}{cleanup}
    and \SubsVar{T}{gossip}.

\subsubsection{Automatic configuration building} \label{subsub:Automagic}

Since building a configuration may be boring, I've written a little
program that does it automatically. The usage is trivial:
\begin{enumerate}
\item   Build a skeleton for the configuration (example in
        \Listing{code:Skeleton});
\item   Crunch and get the configuration for \Erlang\ (example in
        \Listing{code:Automagic}).
\end{enumerate}

\begin{figure}[hbt]
\begin{lstlisting}[label={code:Skeleton},
                   caption={Generating a quick configuration skeleton},
                   language=bash]
$ priv/buildconf.pl -skel quickconf

$ more quickconf
faulty_prob 
beaconwait 
tfail 
tcleanup 
tgossip 
npeers 
deliver_dist 
deliver_mindel 
file_prefix 
deliver_maxdel 
faulty_fail_prob 
keeper 
tbeacon 
statpeers 

\end{lstlisting}
\end{figure}

\begin{figure}[hbt]
\begin{lstlisting}[label={code:Automagic},
                   caption={From quick configuration to \Erlang\
                            \Path{.conf} file},
                   language=]
$ more configs/override

beaconwait 120
deliver_dist {random, uniform, []}
deliver_maxdel 1000
deliver_mindel 100
faulty_fail_prob 0.05
faulty_prob 0.01
file_prefix statfile_default
keeper gfd_keeper
npeers 500
statpeers 0.01
tbeacon 500
tcleanup 20
tfail 10
tgossip 4

$ priv/buildconf.pl < configs/override > override.conf

$ more override.conf
[{ yuna, [
	{faulty_prob, 0.01},
	{faulty_fail_prob, 0.05},
    {deliver_mindel, 500},
	{deliver_maxdel, 1500},
	{deliver_dist, {random, uniform, []}},
	{log_args, { standard_error,
	             "statfile_default" }
	},
	{keeper, gfd_keeper},
	{keeper_args, {{10, 20, 4},
		       500,
		       0.01,
		       500,
		       120}
	}
]}].

$ erl -pa ebin -conf override
Erlang R14B03 (erts-5.8.4) [source] [64-bit] [smp:2:2] [rq:2]
[async-threads:0] [hipe] [kernel-poll:false]

Eshell V5.8.4  (abort with ^G)
1> application:load(yuna).
ok
2> application:get_env(yuna, deliver_mindel).
{ok,100}
3> application:get_env(yuna, deliver_maxdel).
{ok,1000}
\end{lstlisting}
\end{figure}

