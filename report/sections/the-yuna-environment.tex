% -----------------------------------------------------------------------
\subsection{Overview} \label{sub:YunaOverview}
% -----------------------------------------------------------------------

\YUNA\ is a \OTP-compliant \Keyword{application}. It defines the hierarchy
shown in Figure~\ref{pic:YUNA-hier}.

\ImageFW{pictures/yuna}{the \YUNA\ hierarchy}{pic:YUNA-hier}

The involved \Keyword{supervisors} are:
\begin{itemize}

    \item   \Const{main} (file \PathSrc{infrastructure/main.erl}) which is
            placed at the top of the hierarchy, it just starts the
            business logic;

    \item   \Const{services} (file \PathSrc{infrastructure/services.erl})
            which is in charge of managing the permanent services:

    \item   \Const{peers} (file \PathSrc{infrastructure/peers.erl})
            which is a \emph{pool} for the processes which will be
            executing the required distributed logic.

\end{itemize}

The \Keyword{services} under the \Const{services} \Keyword{supervisor}
are:
\begin{itemize}

\item   \Const{logger} (file \PathSrc{infrastructure/log\_serv.erl})
        which provides a simple \Acr{API} for logging events and
        collecting statistics:

    \begin{itemize}

    \item   It implements an \OTP\ \Const{gen\_server};

    \item   As it corresponds to a dedicated process, logging messages are
            enqueued, thus the output text from the processes is never
            overlapped;

    \item   In order to reduce the logging overhead, calls to the logger
            are asynchronous

    \item   It supports many different calls for collecting statistics and
            signaling events, producing some output files which can be
            elaborated by the \emph{gnuplot} software. The files are:
        \begin{itemize}
        \item   \Path{\emph{prefix}\_node\_count} (trace
                the number of nodes in time);
        \item   \Path{\emph{prefix}\_est\_node\_count}
                (trace the approximated number of nodes seen by a subset
                of the running failure detectors in time);
        \item   \Path{\emph{prefix}\_decision\_count}
                (trace the number of nodes which decided some value in
                time);
        \item   \Path{\emph{prefix}\_events} (events to
                be shown in the time diagram).
        \end{itemize}
            where \emph{prefix} is a string depending on
            the configuration file (\see{\Paragraph{subsub:ConfTechParams}}).

    \end{itemize}

\item   \Const{bcast} (file \PathSrc{infrastructure/bcast.erl}) which
        reads messages from the input queue and forwards them to all the
        processes which subscribed the service:
    \begin{itemize}

    \item   Like \Const{logger}, it implements an \OTP\
            \Const{gen\_server};

    \item   The broadcasting also implements a timer which is used to
            serve subscribers with a periodic beacon (this models the
            internal clock of distributed nodes, however they are not
            supposed to consider it as a \emph{global clock}).

    \end{itemize}

\item   \Const{peers\_keeper} is a generic service, namely it's
        constituted by an \Keyword{abstract component}
        (file \PathSrc{infrastructure/behavs/gen\_keeper.erl})
        which can be used to monitor the working distributed algorithm.
        It must be associated with a \Keyword{specific component} which
        implements the supervision part of the distributed algorithm.

\end{itemize}

The processes under the \Const{peers} \Keyword{supervisor} implement the
logic associated with the actual distributed algorithm. As for the
\Const{peers\_keeper}, the running peers are supposed to be
\Keyword{specific components} implementing an \Keyword{abstract component}
(file \PathSrc{infrastructure/behavs/gen\_peer.erl}).

Both \Const{gen\_keeper} and \Const{gen\_peer} export a \Keyword{behavior}
which forces the implementation of some events, constituting an
\emph{event-driven} environment. They are also provided with a dedicated
\Acr{API}, which abstracts away an internal communication protocol.


% -----------------------------------------------------------------------
\subsection{Startup dynamics} \label{sub:StartupDynamics}
% -----------------------------------------------------------------------

At startup the application reads the configuration file
(\Path{ebin/yuna.app}) which is required to contain the name of a module
implementing the \Const{gen\_keeper} \Keyword{behavior}. Let's call it
\Keyword{specific keeper}. When \Const{gen\_keeper} is activated by the
\Const{services} \Keyword{supervisor}, it's parametrized with the name of
the \Keyword{specific keeper} (\see{\Paragraph{subsub:ConfTechParams}}).

The \Acr{API} can now be used by the \Keyword{specific keeper} to spawn
and control the peers, being notified on what's going on as events are
risen.

From the \emph{peer} point of view, the concept is pretty much the same:
the library function used by \Keyword{specific keeper} to spawn nodes must
be parametrized with the number $N$ of required peers, plus the name of a
module implementing the \Keyword{behavior} of the \Const{gen\_peer}
\Keyword{abstract component}. So far $N$ \Keyword{specific peers} are
spawned.

The spawning operation consists, under the hood, in subscribing
dynamically child specifications to the \Const{peers} \OTP\
\Keyword{supervisor}.


% -----------------------------------------------------------------------
\subsection{The communication system} \label{sub:TheCommunicationSystem}
% -----------------------------------------------------------------------

Many theoretical problem concerning distributed systems are based on some
assumptions about the environment in which the nodes execute the business
logic. In this case we assume that:
\begin{itemize}

\item   The nodes may be \emph{faulty} (i.e. they can possibly
        \emph{crash} during the execution of the algorithm);

\item   The communication channel is reliable (no messages are lost),
        but the communication may require a possibly unbounded time.

\end{itemize}

Both assumptions are reflected in the program, with modelizes them by
implementing a communication system which has the following
characteristics:
\begin{itemize}

\item   With a certain probability \PCrash\ the node may be
        killed just before the transmission;

\item   With probability $1 - \PCrash$ the transmission will
        be achieved correctly, although delayed by a random time interval,
        according to a certain probability distribution.

\end{itemize}

The idea of embedding the \emph{syntetic crash} into the transmission is
justified by the facts that:
\begin{itemize}

\item   Transmission is basically mandatory for any reasonable non-trivial
        distributed algorithm;

\item   During transmission a process can influence the outcome of an
        algorithm.

\end{itemize}

Of course this functionality is active only for \Keyword{peers} sending
data to other peers or in broadcast: there's no point in making the
\Keyword{keeper} or one of the internal service crash. Note also that,
in case of broadcasting, the failure probability is \PCrash, as for
unicast messages.


% -----------------------------------------------------------------------
\subsection{Parameters} \label{sub:EnvParams}
% -----------------------------------------------------------------------

Aside from the parameters of the \emph{Consensus Algorithm}
(see \Subsection{sub:ConsParams}), and the strictly technical ones
(see \Subsection{sub:ConfigurationFile}), the \YUNA\ environment has some
parameters which are bound to the distributed system model:

\begin{description}

    \item[Failure probability]:
        as explained each node fails with a certain probability \PCrash\
        during a message transmission. Initially this was designed as an
        event concerning only a fixed set of processes, then I decided to
        make all nodes equally vulnerable to random crashes (this is more
        realistic).

        This choice may be changed in future, but for the moment
        $\PCrash = \PrName{faulty} \cdot \PrName{fc}$. Of course
        $\PrName{faulty}, \PrName{fc} \in [0, 1]$;

    \item[Random delivery delay]:
        as mentioned, the communication can be delayed randomly. The
        parameters required for this functionality are the following:

        \begin{itemize}

        \item   The minimum delay \SubsVar{\delta}{min}, expressed in
                milliseconds;

        \item   The maximum delay \SubsVar{\delta}{max}, expressed
                in milliseconds;

        \item   The probability distribution of the delay
                \SubsVar{D}{delay}, which must be a distribution taking
                values in the interval $[0, 1]$. For instance, if we take
                the \emph{uniform distribution} between 0 and 1, we obtain
                a delay $\delta$ uniformly distributed between
                \SubsVar{\delta}{min} and \SubsVar{\delta}{max}.

        \end{itemize}

    \item[Number of nodes]:
        to be mentioned here for completeness, despite it's trivial, $n =
        \left|\Pi\right|$.

    \item[Beaconing]:
        the period of time between beacons is \SubsVar{T}{beacon},
        expressed in milliseconds.

\end{description}

As for any other parameter, those can be provided to the application by
means of the \emph{configuration file}.
\See{\Subsection{sub:ConfigurationFile}}.
