\Erlang\ is a functional, interpreted and weakly typed language, which
is provided with a built-in process management \Acr{API} based on
message passing. Some interesting features are:
\begin{itemize}
\item   The capability of spawning processes in an easy way
        (concurrency is provided trough real parallelism, and the
        language supports \Acr{SMP});
\item   Processes can send and receive messages asynchronously;
\item   Dependency chains can be built among processes (so that a
        crashing process also tears down all the dependent ones);
\item   Processes can \emph{monitor} other processes, getting notified
        in case of crash;
\item   Applications can be easily deployed over the network, with hot
        swapping of executable code.
\end{itemize}

This capabilities, plus the \emph{side-effects free} design typical
of a functional language, make \Erlang\ particularly well suited for
distributed concurrent applications.

Apart from the language itself, \Erlang\ comes with a stable and tested
standard library named \OTP\ (\OTPa), which provides a generic framework
for services development and a standard way of structuring applications.

This section gives a very brief overview on the language and on how an
\Erlang\ application is structured. The purpose is just giving a better
understanding of the project: for a more detailed description please
refer to the official \Erlang\ documentation\cite{bib:ErlApp}.


\subsection{About the syntax}

The language syntax is simple and straightforward. Naming
conventions determine the semantics of the identifiers. The main data
types are the following:

\begin{description}

    \item[atoms] are system-wide constant literals defined by
        \lstinline{small_caps_identifiers} or
        \lstinline{'single quoted strings'}. Function names, module
        names and boolean values are atoms.

    \item[numbers] are numerical constants;

    \item[tuples] are defined as items of any type, comma-separated and
        \lstinline|{surrounded, by, braces}|;

    \item[lists] are sequences of items of any (possibly mixed) type.
        For instance
        \lstinline{[atoms, plus, 1, number]}; also strings are lists;

    \item[pids] are process identifiers, usually generated trough the
        language \Acr{API} when spawning processes.

\end{description}

More data types allow to work with low-level data streams and native
platform low-level objects.

All \lstinline{CamelCase} identifiers are \emph{variables}, namely
placeholders for items of any type which can be matched. By example:

\begin{lstlisting}
test_function (Value)
    case Value of
        3 -> io:format("I got the value 3\n");
        goat -> io:format("I got the 'goat' atom\n");
        [] -> io:format("I got an empty list\n");
        X when is_list(X) -> io:format("I got a list\n");
        _ -> io:format("No idea...\n")
    end.
\end{lstlisting}

The conventional signature of this function is \Const{test_function/1},
where the value \emph{1} declares the \emph{arity} of the function


\subsection{About \OTP}

The language provides a concept named \Keyword{behaviors}. It resembles the
semantics of \emph{class} in \Tech{Haskell}, and the semantics of
\emph{interface} in \Tech{Java}: a module declaring a certain
\Keyword{behavior} is forced to provide a set of the callback functions,
which can be called by external modules.

This syntactic tool is used by \OTP\ to provide many
\emph{general-purpose} services, which use callback functions of user
modules as stubs. The callbacks, in turn, are supposed to implement the
actual business logic.

Many generic services are available. Among those:
\begin{itemize}
\item   \Const{gen\_server} implements a \emph{client-server} logic;
\item   \Const{gen\_fsm} implements a \emph{finite state machine} logic;
\item   \Const{gen\_event} implements the logic of an
        \emph{event handler}.
\end{itemize}

Moreover, \OTP\ provides a standard way of organizing hierarchically the
application: the generic \Const{supervisor} module. The associated
\Keyword{behavior} requires the implementing module (henceforth
\Keyword{supervisor}) to export an initialization function
(\Const{init/1}), which is supposed to return a list of
\emph{descriptors}; each \emph{descriptor} contains, as a tuple, a set of
data required to spawn the \Keyword{supervised} processes.

\emph{Child Specifications} of \Keyword{supervised} processes can
declare a process as \Keyword{permanent} (gets always be restarted in
case of termination), \Keyword{temporary} (never restarted) and
\Keyword{transient} (restarted only if it terminates abnormally).

Finally, many generic services require, as design pattern, to be started
as process trough a specific function call. This is valid for the
\Const{supervisor} module as well: a \Keyword{supervisor} can manage
\Keyword{supervisors}, thus we can have a \emph{supervision hierarchy}
(henceforth \Keyword{application}).
