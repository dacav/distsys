The \emph{Consensus problem} is a well-known abstract problem concerning
\emph{distributed systems}: many important results relay on it, and many
algorithms solving it are available.

A correct run of a \emph{Consensus algorithms} requires the
\emph{processes} of an asynchronous distributed environment to achieve the
\emph{quorum}, namely to take consistently the same decision. Such
decision can be generalized by the choice of a value for a truth
assignment.

The \YUNA\ project implements a well designed and configurable environment
which follows the typical assumptions we can find in literature. The
program is written in \Erlang\ (\Web{http://erlang.org}), a functional
programming language developed by \Tech{Ericsson}, which is oriented to
concurrent software. It is particularly well-suited for \emph{highly
reliable systems}.

On top of \YUNA\ I implemented one of the many available algorithms for
solving the \emph{Consensus problem}, namely the one proposed by Mostefaoui and
Raynal \cite{bib:Cons}. As this algorithm requires to be equipped with a
\emph{Failure Detector}, I also implemented the basic version of the
Failure Detector Protocol, as proposed by Robbert van Renesse, Yaron
Minsky, and Mark Hayden \cite{bib:FD}. The implementation is based on a
failure detector of class $\diamond S$ (\emph{Eventually Correct})
\cite{bib:QualityFD}.

\subsection{Contents of this document}

\begin{description}

    \item[Section~\ref{sec:erlang-and-otp}] Gives a quick overview on the
        \Erlang\ programming language;

    \item[Section~\ref{sec:the-yuna-environment}]

\end{description}
