\subsection{The Consensus Problem}

The \emph{Consensus problem} is a well-known abstract problem concerning
\emph{distributed systems}: many important results relay on it, and many
algorithms are available.

An algorithm solving the \emph{Consensus problem} requires the
\emph{processes} of a distributed asynchronous environment to achieve the
\emph{quorum}, namely to take consistently the same decision. Such
decision can be generalized by the choice of a value for a truth
assignment.

\subsection{The implementation}

The \YUNA\ project implements one of the many available algorithms for
solving the \emph{Consensus problem}, also providing a well designed and
configurable environment which follows the characteristic assumptions we
can find in literature.

The program is written in \Erlang\ (\Web{http://erlang.org}), a functional
programming language developed by \Tech{Ericsson}, which is oriented to
concurrent software. It is particularly well-suited for \emph{highly
reliable systems}.

\subsection{Contents of this document}

\begin{description}

\item[Section~\ref{sec:erlang-and-otp}] Gives a quick overview on the
    \Erlang\ programming language;

\item[Section~\ref{sec:the-yuna-environment}]

\end{description}
