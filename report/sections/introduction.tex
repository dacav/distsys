The \emph{Consensus problem} is a well-known abstract problem concerning
\emph{distributed systems}: many important results relay on it, and many
algorithms solving it are available.

A correct run of a \emph{Consensus algorithms} requires the
\emph{processes} of an asynchronous distributed environment to achieve the
\emph{quorum} on a certain decision and to select it consistently. Such
decision can be generalized by the choice of a value for a truth
assignment.

The \YUNA\ project implements a well designed and configurable environment
which follows the typical assumptions we can find in literature. The
program is written in \Erlang\ (\Web{http://erlang.org}), a functional
programming language developed by \Tech{Ericsson}, which is oriented to
concurrent software. It is particularly well-suited for \emph{highly
reliable systems}.

On top of \YUNA\ I implemented one of the many available algorithms for
solving the \emph{Consensus problem}, namely the one proposed by
Mostefaoui and Raynal \cite{bib:Cons}. As this algorithm requires to be
equipped with a \emph{Failure Detector}, I also implemented the basic
version of the Failure Detector Protocol, as proposed by Robbert van
Renesse, Yaron Minsky, and Mark Hayden \cite{bib:FD}. The implementation
is based on a failure detector of class $\diamond S$ (\emph{Eventually
Correct}) \cite{bib:QualityFD}.

\subsection{Contents of this document}

\begin{description}

    \item[\Section{sec:ErlangAndOtp}] Gives a quick overview on the
        \Erlang\ programming language;

    \item[\Section{sec:TheYunaEnvironment}] Explains the core of the
        program the communication system and the parameters needed by
        both;

    \item[\Section{sec:ConsensusOverYUNA}] Explains how the
        selected \emph{Consensus} algorithm has been implemented over the
        \YUNA\ platform;

    \item[\Section{sec:Experiments}] Shows a little tutorial on how to
        launch the algorithm, also giving some statistics;

    \item[\Section{sec:YunaApiReference}] Gives a reference on the
        \Acr{API} provided by the \YUNA\ platform.

\end{description}
