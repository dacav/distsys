% ----------------------------------------------------------------------
\subsection{Tutorial: launching the application}
% ----------------------------------------------------------------------

Conventionally, an \Erlang\ \Keyword{application} has a target directory
in which binary object (\Path{.beam}) are pushed during compilation. This
directory is conventionally named \Path{ebin}, and the interpreter must be
aware of this position. So far, in order to launch the application, use
the following command line:
\begin{lstlisting}[language=bash]
$ erl -pa ebin
Erlang R14B03 (erts-5.8.4) [source] [64-bit] [smp:2:2] [rq:2]
[async-threads:0] [hipe] [kernel-poll:false]

Eshell V5.8.4  (abort with ^G)
1>
\end{lstlisting}

Before launching the application it may be necessary to re-compile the
modules:
\begin{lstlisting}[language=bash]
1> make:all([load]).
...
up_to_date
\end{lstlisting}

Then the application can be launched effectively:
\begin{lstlisting}[language=bash]
2> application:start(yuna).
Loading configuration...
Starting YUNA...
Starting main supervisor...
Starting peers pool...
Starting services...
ok
2012/1/4 3:30:59 <0.98.0>  Nodes started: N=500
2012/1/4 3:30:59 <0.98.0>  Preparing protocol...
2012/1/4 3:30:59 <0.98.0>  Nodes started. Letting them know each other...
...
\end{lstlisting}

After some time (depending on the execution) the process will stop:
\begin{lstlisting}[language=bash]
...
2012/1/4 3:32:4 <0.98.0>  Consensus (should) has been reached:
2012/1/4 3:32:4 <0.98.0>    Value false has been selected by 496 nodes
2012/1/4 3:32:4 <0.98.0>  Killing remaining processes...
2012/1/4 3:32:4 <0.98.0>  Terminating.
2012/1/4 3:32:4 <0.98.0>  Keeper finished.
\end{lstlisting}

This will produce the files containing the statistics. The name of the
file will depend, as explained in \Paragraph{subsub:ConfTechParams}, by a
prefix stored in the configuration variable. In this case (the prefix is
\emph{statfile} we have the following files:
\begin{lstlisting}[language=bash]
$ ls statfile_*
statfile_decision_count.log  statfile_est_node_count.log statfile_events.log  statfile_node_count.log
\end{lstlisting}

From this files we can produce easily the graph of the execution by
feeding the \emph{gnuplot} software with the prefix and the configuration
I've put in the \Path{priv} directory:
\begin{lstlisting}[language=bash]
gnuplot -e 'prefix="statfile"' priv/plot_stats.gnuplot
\end{lstlisting}

This will produce a \emph{pdf} file named after the prefix, in this case
\Path{statfile.pdf}. \Figure{fig:RunDefault} shows the result.

\Image{pictures/run-default}{
    A run of the \emph{consensus} algorithm with the default settings.
    The green points (labelled \emph{est\_node\_count}) show the estimated
    number of nodes as provided by the \emph{Failure Detector} of the
    1\% of the nodes. The actual number of nodes is 500 (no nodes crashed
    during the experiment. The \emph{start protocol} and the \emph{end
    protocol} show respectively the startup instant of the
    \emph{consensus} protocol and the instant in which the nodes achieved
    the agreement.
}{1}{fig:RunDefault}


% ----------------------------------------------------------------------
\subsection{Some experimental data}
% ----------------------------------------------------------------------
